


\documentclass[12pt]{article}%{amsart}
\usepackage[a4paper]{anysize}
\usepackage[german,english]{babel}
\usepackage{times}
\usepackage[utf8]{inputenc}
\usepackage{layouts}

\usepackage{amssymb,stackengine,scalerel}

\setlength\parindent{0pt} % einrückungen verschwinden nach paragraphen etc

\usepackage{ifthen}
\newboolean{beweise} 
\setboolean{beweise}{true}
\usepackage{mathtools}
\usepackage{amssymb}
\usepackage{amsmath}
\usepackage{amsthm}
\usepackage{tikz}
\usetikzlibrary{positioning}
\usepackage{subcaption}
\usepackage{caption}



% Festlegen der sachen für theoreme, Sätze, Lemma, Corrolare und  Definitionen
\newtheoremstyle{break}
  {\topsep}{3pt}%{\topsep}%
  {\itshape}{}%
  {\bfseries}{}%
  {\newline}{}%

\newtheoremstyle{examplestyle}% name of the style to be used
  {5mm}% measure of space to leave above the theorem. E.g.: 3pt
  {5mm}% measure of space to leave below the theorem. E.g.: 3pt
  {\slshape}% name of font to use in the body of the theorem
  {6pt}% measure of space to indent
  {\bfseries}% name of head font
  {}%\newline}% punctuation between head and body
  {10mm}% space after theorem head
  {}% Manually specify head
  
\theoremstyle{examplestyle}
\newtheorem{theo}{Theorem}[section]
\theoremstyle{examplestyle}
\newtheorem{satz}[theo]{Satz}%[section]
\theoremstyle{examplestyle}
\newtheorem{lem}[theo]{Lemma}%[section]
\theoremstyle{examplestyle}
\newtheorem{kor}[theo]{Korollar}%[section]
\theoremstyle{examplestyle}
\newtheorem{prop}[theo]{Proposition}%[section]


%\theoremstyle{definition}
\theoremstyle{examplestyle}
\newtheorem{defi}[theo]{Definition}%[section]

% define operators
\DeclareMathOperator{\rkA}{rang}
\DeclareMathOperator{\rkP}{rk}
\DeclareMathOperator{\rkM}{rk}
\DeclareMathOperator{\codim}{codim}




\begin{document}
\selectlanguage{german}
\newcommand{\3}{\ss}

%makros
\newcommand{\s}[1]{\mathcal{#1}}




\ifthenelse{\boolean{beweise}}
{
  
  \begin{center}
    \noindent{\large\bf Bachelorarbeit \\ Titel todo  \\ Sommersemester 2024}\\[1cm]
    \noindent{\large\bf Arbeitstitel:\\ Todo }\\[3cm]
  \end{center}

  \noindent{\bf Name:} Beckmann\\

  \noindent{\bf Vorname:} Nils\\

  \noindent{\bf Studiengang:} Kognitive Informatik\\

  \noindent{\bf Matrikelnummer:} 2658689\\[3cm]

  \noindent{\bf Gutachter:} Frettlöh todo \\


  \noindent{\bf Datum:} 30.09.2024

  \thispagestyle{empty}
  \newpage
  
}{}

\tableofcontents
\thispagestyle{empty}
\newpage

\setcounter{page}{1}

\section{Einleitung}

\section{}

\section{}


\begin{defi}
  Ein einfacher Graph $G=(V,E)$ ist ein Tupel bestehend aus den Knoten $V$ und den Kanten $E$. Hierbei verbindet eine Kannte immer genau
  zwei Knoten miteinander. Die Knoten werden hierbei durch die Nummerierung von $1$ bis $\#V$  benannt ($V=[\#V]$) und eine Kante\\
  $\{i,j\}\in E$ verbindet
  die Knoten $i$ und $j$ miteinander. In Zukunft wird $E(G)$ für $E$ benutzt, um deutlich zu machen, von welchem Graphen die Kanten stammen.
  Außerdem wird $ij$ anstatt $\{i,j\}$ geschrieben.
\end{defi}


  
\begin{defi} die färbung eines graphen mit kanten $[n]$ ist eine Funktion $\kappa: [n]\to \mathbb{P} $. Die Färbung $\kappa$ ist zulässig,
  wenn $\kappa(i) \neq \kappa(j)$ für $ij \in E(G)$. Wenn $q\in \mathbb{P}$, dann sei $\chi_G(q)$ die Anzahl zulässiger Färbungen $\kappa: [n]\to [q] $ von $G$, also die Zahl der zulässigen Färbungen von $G$ wobei die Farben von den Zahlen  $1,2,...,q$ dargestellt werden.
  Die Funktion $\chi_G$ wird das chromatisches Polynom von G genannt.\\
  Wobei $\mathbb{P}=\mathbb{N}\backslash \{0\}$.%todo prüfen ob ich $\mathbb{P}$ brauche oder auch $\mathbb{N}$ geht 
\end{defi}

Notiz: Es muss gezeigt werden, dass $\chi_G(q)$ ein Polynom ist. Hierfür siehe Seite 24-25 in \cite{stan}


wad sdfsf sf .
\begin{kor}[Nils Beckmann]

  
\end{kor}











 Randnotiz: Alternativ können Matroide auch über ihre Kreise oder Basen (mit anderen Axiomen) definiert werden (vgl. \cite{MatBuch}).

\section{Das charakteristische Polynom aus der Perspektive von Matrioiden und Halbmatroiden}

% \ref{semiEigenschaften} 

\begin{lem}\label{C-e=C/e}
 
\end{lem}
\begin{proof}

\end{proof}





\begin{defi}
  
\end{defi}


\begin{prop}[Tutte-Grothedieck Invariante für das Tutte Polynom]\label{T-GInv}
 
\end{prop}

\begin{proof}
 
\end{proof}


\section{Fazit}




\newpage
\thispagestyle{empty}

% zitieren wie folgt: \cite{stan}
\begin{thebibliography}{9}
\bibitem{stan}
  Richard P. Stanley (2007) \emph{An Introduction to Hyperplane Arrangements}, IAS/Park City Mathematics Series, vol. 13, American Mathematical Society

  %enu_comb_stanley.pdf
\bibitem{enum}
  Richard P. Stanley(1996), \emph{Enumerative Combinatorics}, vol. 1, Wadsworth and Brooks/Cole,
Pacific Grove, CA, 1986; second printing, Cambridge University Press 

\bibitem{orlik}
  Peter Orlik und Hiroaki Terao (1992) \emph{Arrangemnts of Hyperplane}, Springer

\bibitem{SHI/ISH}
  Drew Armstrong and Brendon Rhoades, \emph{THE SHI ARRANGEMENT AND THE ISH ARRANGEMENT}, 2010, Zugriff am 01.10.2023,
  arXiv:1009.1655

\bibitem{MIT}
  Ardila, Federico, \emph{Enumerative and algebraic aspects of matroids and hyperplane arrangements}, 2003, Zugriff am 15.09.2023,
  https://dspace.mit.edu/handle/1721.1/29287

\bibitem{Linial}
  Richard P Stanley(1996) \emph{Hyperplane arrangements, interval orders, and trees}, Zugriff am 17.10.2023
  https://www.pnas.org/doi/abs/10.1073/pnas.93.6.2620
  %https://scholar.google.de/citations?view\_op=view\_citation\&hl=de\&user=Rc5HbpQAAAAJ\&cstart=100\&pagesize=100\&sortby=pubdate\&citation\_for\_view=Rc5HbpQAAAAJ:yqoGN6RLRZoC

\bibitem{Zas}
  Thomas Zaslavsky(1975) \emph{Facing up to Arrangements: Face-Count Formulas for Partitions of Space by Hyperplanes}
  eBook ISBN: 978-0-8218-9955-7, Memoirs of the American Mathematical Society

\bibitem{Atha}
  Christos A. Athanasiadis(1996), \emph{Characteristic Polynomials of Subspace Arrangements and Finite Fields},
  advances in mathematics 122 S.193-233,
  Zugriff am 20.10.2023, https://doi.org/10.1006/aima.1996.0059 

\bibitem{MatBuch}
  James Oxley (2011) \emph{Matroid Theory} zweite Edition, Oxford University Press Inc., New York,
  ISBN 978–0–19–856694–6

\bibitem{WhatMat}
  James Oxley \emph{BRIEFLY, WHAT IS A MATROID?},Zugriff am 10.09.2023
  https://www.math.lsu.edu/\raisebox{-0.9ex}{\~{}}oxley/matroid\_intro\_summ.pdf

\bibitem{CoxArr}
  Christos A. Athanasiadis(1998)\emph{Deformations of Coxeter hyperplane arrangements and their characteristic polynomials},
  Advanced Studies in Pure Mathematics 27, Zugriff am 21.10.2023 auf Google scholar
  
\end{thebibliography}



\end{document}